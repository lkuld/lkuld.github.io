\documentclass[xcolor=dvipsnames,10pt]{beamer}

\usepackage[utf8]{inputenc}
\usepackage{amssymb}

\graphicspath{ {Graphs/} }
 \usepackage[ngerman]{babel}


\usepackage{lmodern}% http://ctan.org/pkg/lm
\usepackage{adjustbox}
\usepackage{tikz}
\usetikzlibrary{arrows.meta}

\usepackage{amsmath}
\beamertemplatenavigationsymbolsempty

\newlength{\wideitemsep}
\setlength{\wideitemsep}{\itemsep}
\addtolength{\wideitemsep}{10pt}
\let\olditem\item
\renewcommand{\item}{\setlength{\itemsep}{\wideitemsep}\olditem}


\title[Standortstruktur]{Standortstruktur und Regionalentwicklung}
\subtitle{Budgetgerade und Indifferenzkurve}
\institute{URI, Wiwi, TU Dortmund}
\author[]{{Lukas Kuld\thanks{\tiny{\href{mailto:lukas.kuld@udo.edu}{lukas.kuld@udo.edu}}}}}              
\date{\today}

\setbeamertemplate{footline}[frame number]


\begin{document}
\frame{
	\titlepage
}

\frame{
	\frametitle{Indifferenzkurven}
	\begin{minipage}[b]{0.5\linewidth}
		\begin{figure}[!h]
\begin{tikzpicture}[scale=0.8,>=Triangle,thick]
\draw[->](0,0)node[below]{$0$}--(6,0)node[below left]{Wohnraum q};
\draw[->](0,0)--(0,6)node[below left]{Rest c};

\draw (0.2,5.3) to [bend right=40]  coordinate[pos=0.2] (l_1)(5,0.5);
\draw (l_1)node[above right]{$u_1$};
\draw (1.2,5.3) to [bend right=40]  coordinate[pos=0.2] (l_2)(5,1.5);
\draw (l_2)node[above right]{$u_2$};
\draw (2.2,5.3) to [bend right=40]  coordinate[pos=0.2] (l_3)(5,2.5);
\draw (l_3)node[above right]{$u_3$};

\end{tikzpicture}
\end{figure}
	\end{minipage}
	\hspace{0.5cm}
	\begin{minipage}[b]{0.4\linewidth}
		\begin{itemize}
			\item Haushalt zieht aus dem Konsum zweier Güter Nutzen
			\item Je höher der Konsum an Wohnraum $q$ und Rest $c$ ist, desto höher der Nutzen
			\item Also: $u_3 > u_2 > u_1$
			\item Entlang den Kurven gibt jeder Punkt (Konsumbündel) den gleichen Nutzen
		\end{itemize}
	\end{minipage}
}


\frame{
	\frametitle{Budgetgerade I}
	\begin{minipage}[b]{0.5\linewidth}
		\begin{figure}[!h]
\begin{tikzpicture}[scale=0.8,>=Triangle,thick]
\draw[->](0,0)node[below]{$0$}--(6,0)node[below left]{Wohnraum q};
\draw[->](0,0)--(0,6)node[below left]{Rest c};

\draw (0,5.3) to  coordinate[pos=0.8] (l_1)(5,0);
\draw (l_1)node[left]{$B_3$};
\draw (0,3.8) to   coordinate[pos=0.8] (l_2)(3.5,0);
\draw (l_2)node[left]{$B_2$};
\draw (0,2.3) to  coordinate[pos=0.8] (l_3)(2,0);
\draw (l_3)node[left]{$B_1$};

\end{tikzpicture}
\end{figure}
	\end{minipage}
	\hspace{0.5cm}
	\begin{minipage}[b]{0.4\linewidth}
		\begin{itemize}
			\item Haushalt hat ein Budget aus dem verfügbaren Einkommen, z.B. $y-tx$
			\item Kann davon Wohnraum $q$ zum Preis $p_q$ erwerben oder Rest $c$ zum Preis $p_c$ (in der VL $p_c$=1)
			\item Gibt je nach $q$ und $c$ also $p_q q + p_c c$ aus
		\end{itemize}
	\end{minipage}
}

\frame{
	\frametitle{Budgetgerade II}
	\begin{minipage}[b]{0.5\linewidth}
		\begin{figure}[!h]
\begin{tikzpicture}[scale=0.8,>=Triangle,thick]
\draw[->](0,0)node[below]{$0$}--(6,0)node[below left]{Wohnraum q};
\draw[->](0,0)--(0,6)node[below left]{Rest c};

\draw (0,5.3) to  coordinate[pos=0.8] (l_1)(5,0);
\draw (l_1)node[left]{$B_3$};
\draw (0,3.8) to   coordinate[pos=0.8] (l_2)(3.5,0);
\draw (l_2)node[left]{$B_2$};
\draw (0,2.3) to  coordinate[pos=0.8] (l_3)(2,0);
\draw (l_3)node[left]{$B_1$};

\end{tikzpicture}
\end{figure}
	\end{minipage}
	\hspace{0.5cm}
	\begin{minipage}[b]{0.4\linewidth}
		\begin{itemize}
			\item Bei ausschöpfen des Budgets: $y-tx = p_q q + p_c c$ 
			\item oder $c = \frac{y-tx}{p_c} - \frac{p_q}{p_c} q$
			\item $\frac{y-tx}{p_c}$ ist also der y-Achsenabschnitt
			\item und $- \frac{p_q}{p_c}$ die Steigung der Budgetgeraden
		\end{itemize}
	\end{minipage}
}



\frame{
	\frametitle{Budgetgerade III}
	\begin{minipage}[b]{0.5\linewidth}
		\begin{figure}[!h]
\begin{tikzpicture}[scale=0.8,>=Triangle,thick]
\draw[->](0,0)node[below]{$0$}--(6,0)node[below left]{Wohnraum q};
\draw[->](0,0)--(0,6)node[below left]{Rest c};

\draw (0,5.3) to  coordinate[pos=0.8] (l_1)(5,0);
\draw (l_1)node[left]{$B_2$};
\draw (0,3.8) to   coordinate[pos=0.8] (l_2)(3.5,0);
\draw (l_2)node[left]{$B_1$};

\end{tikzpicture}
\end{figure}
	\end{minipage}
	\hspace{0.5cm}
	\begin{minipage}[b]{0.4\linewidth}
		\begin{itemize}
			\item Wie erhöht sich die Budgetgerade $c = \frac{y-tx}{p_c} - \frac{p_q}{p_c} q$ 
			\item Angenommen $y \uparrow$ oder $t \downarrow$, dann $\frac{y-tx}{p_c} \uparrow$
			\item Ergibt eine Parallelverschiebung wie im Bild
			\item Gleiches gilt für eine Änderung der Preise solange diese relativ gleich bleiben, d.h. $\frac{p_q}{p_c}$ unverändert bleibt
		\end{itemize}
	\end{minipage}
}

\frame{
	\frametitle{Budgetgerade IV}
	\begin{minipage}[b]{0.5\linewidth}
		\begin{figure}[!h]
\begin{tikzpicture}[scale=0.8,>=Triangle,thick]
\draw[->](0,0)node[below]{$0$}--(6,0)node[below left]{Wohnraum q};
\draw[->](0,0)--(0,6)node[below left]{Rest c};

\draw (0,5.3) to  coordinate[pos=0.8] (l_1)(5,0);
\draw (l_1)node[left]{$B_2$};
\draw (0,5.3) to   coordinate[pos=0.8] (l_2)(3.5,0);
\draw (l_2)node[left]{$B_1$};

\end{tikzpicture}
\end{figure}
	\end{minipage}
	\hspace{0.5cm}
	\begin{minipage}[b]{0.4\linewidth}
		\begin{itemize}
			\item Falls sich das Preisverhältnis ändert, z.B. weil sich nur ein Preis ändert, ändert sich die Steigung der Budget-Geraden 
			\item Im Bild ist Wohnraum günstiger geworden ($p_q \downarrow$) und der Preis für Restkonsum unverändert, daher $- \frac{p_q}{p_c} \uparrow$ 
		\end{itemize}
	\end{minipage}
}

\frame{
	\frametitle{Budgetgerade V}
	\begin{minipage}[b]{0.5\linewidth}
		\begin{figure}[!h]
\begin{tikzpicture}[scale=0.8,>=Triangle,thick]
\draw[->](0,0)node[below]{$0$}--(6,0)node[below left]{Wohnraum q};
\draw[->](0,0)--(0,6)node[below left]{Rest c};

\draw (0,3.5) to  coordinate[pos=0.8] (l_1)(5.3,0);
\draw (l_1)node[left]{$B_2$};
\draw (0,5.3) to   coordinate[pos=0.8] (l_2)(3.5,0);
\draw (l_2)node[left]{$B_1$};

\end{tikzpicture}
\end{figure}
	\end{minipage}
	\hspace{0.5cm}
	\begin{minipage}[b]{0.4\linewidth}
		\begin{itemize}
			\item Bei unterschiedlichen Entfernungen zum ZGB, beobachten wir in unserem Stadtmodell eine Einkommensverschiebung und eine Preisanpassung
			\item Zuerst verringert sich das verfügbare Einkommen, wenn wir weiter pendeln müssen: $y-tx \downarrow$
			\item Dann passt sich der Quadratmeterpreis an, damit der Nutzen konstant bleibt: $p_q \uparrow$
		\end{itemize}
	\end{minipage}
}


\frame{
	\frametitle{Budgetgerade und Indifferenzkurve I}
	\begin{minipage}[b]{0.5\linewidth}
		\begin{figure}[!h]
\begin{tikzpicture}[scale=0.8,>=Triangle,thick]
\draw[->](0,0)node[below]{$0$}--(6,0)node[below left]{Wohnraum q};
\draw[->](0,0)--(0,6)node[below left]{Rest c};


\draw (0,3.8) to   coordinate[pos=0.45] (l_2)(3.5,0);
\fill (l_2)circle(2.2pt)node[left]{$(q_0,c_0)$};
\draw (1.2,5.3) to [bend right=40]  coordinate[pos=0.2] (l_2)(5,1.5);


\draw (0.2,5.3) to [bend right=40]  coordinate[] (l_1)(5,0.5);


\end{tikzpicture}
\end{figure}
	\end{minipage}
	\hspace{0.5cm}
	\begin{minipage}[b]{0.4\linewidth}
		\begin{itemize}
			\item Gegeben die gezeigt Budgetgerade, erreicht der Haushalt im Punkt $(q_0,c_0)$ den höchsten Nutzen
		\end{itemize}
	\end{minipage}
}



\frame{
	\frametitle{Budgetgerade und Indifferenzkurve II}
	\begin{minipage}[b]{0.5\linewidth}
		\begin{figure}[!h]
\begin{tikzpicture}[scale=0.8,>=Triangle,thick]
\draw[->](0,0)node[below]{$0$}--(6,0)node[below left]{Wohnraum q};
\draw[->](0,0)--(0,6)node[below left]{Rest c};

\draw (0,2.5) to  coordinate[pos=0.7] (l_1)(5.3,0);
\fill (l_1)circle(2.2pt)node[above right]{$(q_1,c_1)$};
\draw (0,5.3) to   coordinate[pos=0.3] (l_2)(2.5,0);
\fill (l_2)circle(2.2pt)node[above right]{$(q_0,c_0)$};

\draw (0.32,5) to [bend right=30]  coordinate[pos=0.2] (l_2)(5,0.32);

\end{tikzpicture}
\end{figure}
	\end{minipage}
	\hspace{0.5cm}
	\begin{minipage}[b]{0.4\linewidth}
		\begin{itemize}
			\item Bei unterschiedlichen Entfernungen zum ZGB, beobachten wir in unserem Stadtmodell eine Einkommensverschiebung und eine Preisanpassung
			\item Zuerst verringert sich das verfügbare Einkommen, wenn wir weiter pendeln müssen: $y-tx \downarrow$
			\item Dann passt sich der Quadratmeterpreis an, damit der Nutzen konstant bleibt: $p_q \uparrow$
		\end{itemize}
	\end{minipage}
}



\end{document}
